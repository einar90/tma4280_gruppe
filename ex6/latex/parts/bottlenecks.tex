% -*- root: ../ex6.tex -*-

\clearpage
\section{Bottlenecks and optimization} % (fold)
\label{sec:bottlenecks_and_optimization}

In our solver the root prosess has a lot of responsibilities. It is responsible for initializing the data, distributing it across all the other prosesses, collecting the data, and doing all of the transposes. This introduces a potential bottleneck in the solver depending on how large a problem size you want to run on it.

Because all the network flow has to pass through our root prosess latency is expected to be larger than if every prosess arranged its own data for a collective distribution to all the other prosesses through \texttt{MPI\_Alltoallv}. Our reason for doing it this way is that we felt this approach had less complexity involved, and it was still possible to achieve a reasonable speedup. We timed each part of the serial version of \texttt{poission2D} and saw that the transpose operation used minimal time compoared to \emph{FST}.

There are two possible improvements that can be done here. If you need to be able to solve larger tasks it would be advised to do the transpose in the \texttt{MPI\_Send/MPI\_Recv} itself. This could be achieved by the sender passing its columns to the reciever, and the reciever saving them as rows in its own matrix. As every prosessor has access to the \texttt{displacements} array it could easily place the recieved data from the other prosesses in the right place. This solution expects that all the recievers have a full $(N-1)\times(N-1)$ matrix in memory to be able to recieve whole rows/columns. The method is illustrated in Figure~\ref{fig:send_transpose}. In our solver, this would be the easiest change to make for a better speedup as the root prosess could save the recieved columns as rows.

The other possible improvement is to make every prosess responsible for distributing its own data to all the other prosesses. As every prosess has access to the \texttt{partlens} they would all be aware of how much data is going to be sent/recieved by all the other prosesses. This could be implemented in the same fashion as mentioned earlier with every prosess having its own matrix in memory, or by a more complex solution where every process only recieves the data it expects (parts of columns from other prosesses) and nothing else, but this requires more complex code. The more complex solution would require less memory, as every prosess would not need to have the full matrix in memory. This would also enable us to include the root prosess in more calculations, as it no longer has to handle all the data transfers.

\begin{figure}[htbp]
  \centering
  \includegraphics[width=\textwidth]{illustrations/matrix_blocktranspose.pdf}
  \caption{The transpose operation using message passing: the packaging and unpackaging of data. The figure is due to Bjarte Hægland.}
  \label{fig:send_transpose}
\end{figure}








% subsection bottlenecks_and_optimization (end)

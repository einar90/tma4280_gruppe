% -*- root: ../ex6.tex -*-

\section{Discretizing the Poisson problem} % (fold)
\label{sec:discretizing_the_poisson_problem}
Considering the solution of the two-dimensional Poisson problem:
\begin{align}
  - \nabla u &= f \;\;\; \text{in } \Omega = (0,1) \times (0,1) \label{eq:poisson2d} \\
  u &= 0 \;\;\; \text{on } \partial \Omega \label{eq:poisson2d_edge}
\end{align}
Expanding equation \eqref{eq:poisson2d}:
\begin{equation}
  -u_{xx} - u_{yy} = f(x,y) \label{eq:poisson2d_expanded}
\end{equation}
Discretizing equation \eqref{eq:poisson2d_expanded} using a mesh like the one illustrated in Figure~\ref{fig:mesh} standard 5-point stencil illustrated in Figure~\ref{fig:stencil}:
\begin{equation}
  \nonumber
  - \frac{u_{i-1,j} - 2u_{i,j} + u_{i+1,j}}{h^2} - \frac{u_{i,j-1} - 2u_{i,j} + u_{i,j+1}}{h^2} = f_{i,j}
\end{equation}
\begin{equation}
  -u_{i-1,j} -u_{i,j-1} + 4u_{i,j} - u_{i+1,j} - u_{i,j+1} = h^2 \cdot f_{i,j}
\end{equation}
This equation, combined with a numbering scheme like the one illustrated in Figure~\ref{fig:numbering_scheme} gives us a penta-diagonal matrix equation $\mathbf{Au}=h^2\mathbf{f}$ :
\begin{equation}
  \begin{bmatrix}
    4  & -1 & -1 & 0  & 0  & 0  & \cdots & 0 & 0 \\
    -1 & 4  & -1 & -1 & 0  & 0  & \cdots & 0 & 0 \\
    -1 & -1 & 4  & -1 & -1 & 0  & \cdots & 0 & 0 \\
    0  & -1 & -1 & 4  & -1 & -1 & \cdots & 0 & 0 \\
       & & \ddots & \ddots & \ddots &   &        &   &   \\
    0   & 0   & \cdots   &  0  & 0  &  0  &  -1  & -1 & 4
  \end{bmatrix}
  \times
  \begin{bmatrix}
    u_0 \\ u_1 \\ u_2 \\ \vdots \\ u_{n-1} \\ u_n
  \end{bmatrix}
  =
  h^2 \cdot
  \begin{bmatrix}
    f_0 \\ f_1 \\ f_2 \\ \vdots \\ f_{n-1} \\ f_n
  \end{bmatrix}
\end{equation}


\begin{figure}[htbp]
  \centering
  \includegraphics{illustrations/mesh.pdf}
  \caption{An illustration of a general two-dimensional mesh.}
  \label{fig:mesh}
\end{figure}

\begin{figure}[htbp]
  \centering
  \includegraphics[]{illustrations/stencils.pdf}
  \caption{Illustration of the standard 5-point stencil.}
  \label{fig:stencil}
\end{figure}

\begin{figure}[htbp]
  \centering
  \includegraphics{illustrations/equation_numbering.pdf}
  \caption{The numbering scheme used.}
  \label{fig:numbering_scheme}
\end{figure}


% section discretizing_the_poisson_problem (end)

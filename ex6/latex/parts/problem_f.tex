% -*- root: ../ex6.tex -*-
\subsection{Non-homogeneous Dirichlet boundary conditions}
So far we have considered the case where $u_{\partial \Omega} = 0$. Having a constant value on the boundary
\begin{equation}
  u_{\partial \Omega} = C_{\partial \Omega}
\end{equation}
is just as simple: solve
\begin{equation}
-\nabla^2 \hat{u} = 0,
\end{equation}
then substitute back
\begin{equation}
u = \hat{u} + C_{\partial \Omega}.
\end{equation}

When the boundary is not constant, this gets more tricky. Using the discrete sine transform (DST) to obtain eigenvalues and eigenvectors depends on u having the same boundary conditions as the sine function in use. Consider the 1D case, where $u = u(x)$:
\begin{equation}
	u(0) = u(1) = 0.
\end{equation}
This gives us a series of sine functions that we can use to create an odd fourier series to represent our $u$,
\begin{equation}
	s_j(x) = sin\left(j \pi x\right)
\end{equation}
because
\begin{equation}
	s_j(0) = s_j(1) = 0, \;\;\; j \in \mathbb{N}.
\end{equation}
In other words: \emph{no}, we cannot use DST for arbitrary boundary conditions. This is not a limitation of a general fourier sum
\begin{equation}
	f(x) = \frac{1}{2} a_0
		+ \sum_{j=1}^{\infty} a_j \cos(j \pi x)
		+ \sum_{j=1}^{\infty} b_j \sin(j \pi x)
		,
\end{equation}
which is in fact the common way to solve Navier's thin plate problem
\begin{equation}
-\nabla^4 w = -\frac{q}{D}.
\end{equation}
... wich is not too different from the poisson equation. Digression aside: we have not explored whether fourier series may be used to eigenvalues for general Dirichlet boundary conditions, but it does not have the obvious limitation DST has.

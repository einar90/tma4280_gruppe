% -*- root: ../ex6.tex -*-
\section{Performance} % (fold)
\label{sec:performance}

\subsection{Comparison of MPI and OpenMP performance} % (fold)
\label{sub:comparison_of_mpi_and_openmp_performance}
In this section we compare the run times for $n=16,384$ and various combinations of the number of MPI processes $p$ and OpenMP threads pr. process $t$, where $p\times t = 36$. The results are shown in Table~\ref{tab:runtimes_36}. In our case, we achieve the best performance for large $t$ and low $p$: the run time with $t=36, p=1$ is 16.3\% lower than the second best run time, with $t=18, p=2$. In the mid range we find the run times for high numbers of $p$ and low numbers of $t$. We get the highest run times for close numbers of $t$ and $p$.

The significantly lower run time for $p=1$ is likely due to the absence of \texttt{MPI\_Send/Recv} calls, which introduces a significant latency in the program.

The relatively good run time for $p=36$ is explained by the fact that one node is ``reserved'' for transposing and distributing the matrix, while the 35 others are responsible for transformations. This means 35 transformations can run in parallel in this case, while in the cases with $p=18,12,9,6$ only $(p-1)\times t = 34, 33, 32$ and $30$ threads, respectively, run in parallel.

Note that the run time for $p=1,2$ was acquired from runs on one Kongull node, while the others were acquired from runs on three nodes. This is likely the explanation for the low run time for $p=2, t=1$, but it should not affect the case with $p=1,t=36$ as the processes should not ``spread'' to the other nodes when MPI is not active.

In conclusion: out program does not utilize the hybrid model very well at all.

\begin{table}[H]
  \centering
  \caption{Measured runtimes when $p\times t = 36$, for $n=16,386$.}
  \label{tab:runtimes_36}
  \begin{tabularx}{0.5\textwidth}{XXXX}
    \toprule
    $t$ & $p$ & $\tau$ \\
    \midrule
    36  &  1  &  112.515614 \\
    18  &  2  &  134.508309 \\
    1   & 36  &  136.474492 \\
    2   & 18  &  141.108449 \\
    3   & 12  &  143.360202 \\
    4   &  9  &  147.304511 \\
    6   &  6  &  153.733214 \\
    \bottomrule
  \end{tabularx}
\end{table}

% subsection comparison_of_mpi_and_openmp_performance (end)
% section performance (end)

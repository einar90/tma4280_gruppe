% -*- root: ../ex6.tex -*-

\subsection{Speedup and parallell efficiency} % (fold)
\label{sub:speedup_and_parallell_efficiency}
In this section we discuss the speedup\footnote{A measure of how much the run time is reduced when the number of processors increases.} $S_p$  and parallel efficiency\footnote{A measure of how well the program utilizes parallel resources.} $\eta_p$ for our program, where
\begin{equation}
  S_p = \frac{\tau_1}{\tau_p}
\end{equation}
\begin{equation}
  \eta_p = \frac{S_p}{P}
\end{equation}
where $\tau_p$ denotes the measured run time on $p$ processors. The results are shown in Figure~\ref{fig:plot_pe_sp} and Table~\ref{tab:speedup_all_data}. From the plot/table we see that our program achieves a maximum of 5.82 for $P\times T$ = 36, which is far less than the optimal speedup for 36 processors (which would be $S_p=36$).

We also see from the $\eta_p$ lines in Figure~\ref{fig:plot_pe_sp} that our program does not utilize parallell resources very well: it drops drastically with an increasing number of processors, and at our point optimal speedup $\eta_p=0.16$ which is not very good.

When in comes to speedup in relation to the problem size $n$, we see that increasing the problem size from 8192 to 16384 does improve the speedup. This is due to the increased benefit of parallelizing the transform.


\begin{figure}[htbp]
  \centering
  \includegraphics[width=\textwidth]{illustrations/plots/speedup_parallell_efficienct_combo.pdf}
  \caption{Combination plot of parallell efficiency and speedup. Only the best runtimes for equal $p\times t$ are plotted. The complete dataset is shown in Table~\ref{tab:speedup_all_data}.}
  \label{fig:plot_pe_sp}
\end{figure}

\begin{table}[H]
  \centering
  \caption{All measured datapoints for $N=16,384$ and various numbers of $t$ and $p$.}
  \label{tab:speedup_all_data}
  \begin{tabularx}{0.9\textwidth}{XXXX|XX}
    \toprule
    $P\times T$ & $t$ & $p$ & $\tau$ & $S_p$ & $\eta_p$ \\
    \midrule
    36   &  1   & 36    &  136.474492 & 5.82  & 0.16\\
    36   &  2   & 18    &  141.108449 & 5.63  & 0.16\\
    36   &  3   & 12    &  143.360202 & 5.54  & 0.15\\
    36   &  4   & 9     &  147.304511 & 5.39  & 0.15\\
    36   &  6   & 6     &  153.733214 & 5.17  & 0.14\\
    24   &  4   & 6     &  154.214797 & 5.15  & 0.21\\
    18   &  3   & 6     &  162.046432 & 4.90  & 0.27\\
    \bottomrule
  \end{tabularx}
\end{table}

% subsection speedup_and_parallell_efficiency (end)

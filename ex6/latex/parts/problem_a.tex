% -*- root: ../ex6.tex -*-

\section{Implementation} % (fold)
\label{sec:implementation}
The program written for this exercise was written in \emph{C}. It is based on the \texttt{poisson2D} program given as an example in the course GitHub repository\footnote{\url{https://github.com/akva2/tma4280}}. An illustration of the parallellized program flow is given in Figure~\ref{fig:program_flow}. A more detailed description of the program flow follows:
\begin{description}
  \item [MPI check:] The program checks if the set number of MPI processes is equal to 1 or if MPI is disabled. If it \emph{is} a separate version of the diagonalization function with no MPI calls is called (\texttt{DiagonalizationPoisson2DfstSerial}). If it is \emph{not}, the \texttt{DiagonalizationPoisson2Dfst} function -- utilizing both MPI and OpenMP -- is called.
  \item [Initialization:] The root process initializes various matrices; the other processes only initialize the $b$ matrix.
  \item [Diagonalization:] All processes call the \texttt{DiagonalizationPoisson2Dfst}.
  \begin{enumerate}
    \item Process 0 distributes the $b$ matrix evenly among the other processes.
    \item After receiving their parts, each of the other MPI-process calls the \texttt{fst} function in an OpenMP parallellized for-loop. The buffer is freed before and after each call to \texttt{fst}. The separate parts are then sent back to the root MPI process.
    \item After receiving the parts and reassembling the now transformed $b$ matrix, process 0 transposes the matrix before dividing and distributing it among the other processes again.
    \item After receiving the parts, each of the other processes calls the \texttt{ifst} function in an OpenMP parallellized for loop. The buffer is freed before and after each call to \texttt{fst}. The separate parts are then sent back to the root MPI process.
    \item After receiving and reassembling the parts again, now in the $ut$ matrix, process 0 divides the elements in the matrix by the $\lambda$ values.
    \item The next steps are identical to steps 1-4. After this we have the solution.
  \end{enumerate}
\end{description}

\begin{figure}[htbp]
  \centering
  \includegraphics[width=\textwidth]{illustrations/parallell_program_flow.pdf}
  \caption{An illustration of the parallellized program flow. $\blacksquare$ indicates the beginning and end of the program; $\bullet$ indicates \texttt{MPI\_Send}/\texttt{MPI\_Recv}; $S$ indicates \emph{Fast Sine Transform}; $S^{-1}$ indicates \emph{inverse Fast Sine Transform}; $T$ indicates matrix transpose.}
  \label{fig:program_flow}
\end{figure}


% section implementation (end)

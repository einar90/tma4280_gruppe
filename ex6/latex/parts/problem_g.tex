% -*- root: ../ex6.tex -*-

\subsection{Solving on larger domains}
This far, we have found solutions for the unit square $(0, 1) \times (0,1)$. We will now consider a general rectangular surface $(0, L_x) \times (0,L_y)$. This gives new finite differences. Given division into $n+1$ subintervals in both directions, we have
% Introduce the dimensionless variables $\hat{x}$ and $\hat{y}$:

% \begin{align}
% \hat{x} &= \frac{x}{L_x} \\
% \hat{y} &= \frac{y}{L_y}
% \end{align}

% Calculate the new differentials:

% \begin{align}
% \frac{\partial}{\partial x} &= \frac{1}{L_x} \frac{\partial}{\partial \hat{x}} \\
% \frac{\partial}{\partial y} &= \frac{1}{L_y} \frac{\partial}{\partial \hat{y}}
% \end{align}

% And insert these into the Poisson equation gives

% \begin{equation}
% -\left(
% 	\frac{1}{L_x^2} \frac{\partial^2}{{\partial \hat{x}}^2}
% 	+ \frac{1}{L_y^2} \frac{\partial^2}{{\partial \hat{y}}^2}
% \right) u = f(x,y)
% \end{equation}

\begin{align}
h_x &= \frac{L_x}{n} \\
h_y &= \frac{L_y}{n}
\end{align}

We use the sine solution strategy as described by the lecture notes. We define the matrices $\mathbf{U}$ and $\mathbf{T}$ as

\begin{equation}
  \mathbf{{U}} = 
  \begin{bmatrix}
    u_{1,1} & \ldots & \ldots & u_{1,n-1} \\
    \vdots & & & \vdots \\
    \vdots & & & \vdots \\
    u_{n-1,1} & \ldots & \ldots & u_{n-1,n-1}
  \end{bmatrix}
\end{equation}
\begin{equation}
  \mathbf{{T}} = 
  \begin{bmatrix}
    2 & -1 & & & 0 \\
    -1 & 2 & -1 & & \\
    & & \ddots & & \\
    & & -1 & 2 & -1 \\
    0 & & & -1 & 2
  \end{bmatrix}.
\end{equation}

Premultiplying $\mathbf{U}$ with $\mathbf{T}$ gives

\begin{alignat*}{2}
  (\mathbf{{T}} \, \mathbf{{U}})_{ij} &= 2u_{i,j} - u_{i+1,j}, &\qquad &i=1, \\
  (\mathbf{{T}} \, \mathbf{{U}})_{ij} &= -u_{i-1,j}+2u_{i,j} - u_{i+1,j}, &\qquad 2 \leq &i \leq n-2, \\
  (\mathbf{{T}} \, \mathbf{{U}})_{ij} &= -u_{i-1,j}+2u_{i,j}, &\qquad &i=n-1.
\end{alignat*}

Analogously, postmultiplying with $\mathbf{T}$ gives

\begin{alignat*}{2}
  (\mathbf{{U}} \, \mathbf{{T}})_{ij} &= 2u_{i,j} - u_{i,j+1}, &\qquad &i=1, \\
  (\mathbf{{U}} \, \mathbf{{T}})_{ij} &= -u_{i,j-1}+2u_{i,j} - u_{i,j+1}, &\qquad 2 \leq &i \leq n-2, \\
  (\mathbf{{U}} \, \mathbf{{T}})_{ij} &= -u_{i,j-1}+2u_{i,j}, &\qquad &i=n-1.
\end{alignat*}

We observe that, when using the five point stencil, 

\begin{align}
  \frac{1}{h_x^2} (\mathbf{{T}} \, \mathbf{{U}})_{ij} \approx -\left( \frac{\partial^2 u}{\partial x^2} \right)_{i,j}
\end{align}

\begin{align}
  \frac{1}{h_y^2} (\mathbf{{U}} \, \mathbf{{T}})_{ij} \approx -\left( \frac{\partial^2 u}{\partial y^2} \right)_{i,j}
\end{align}

\ldots wich differs from the expressions we have on the unit square. We may now use these approximations for discretization:

\begin{equation}
\begin{array}{ccl}
	-\frac{\partial^2 }{\partial x^2} u &-\frac{\partial^2}{\partial y^2} u &= f \\
  \frac{1}{h_x^2}(\mathbf{{T}} \, \mathbf{{U}})_{ij} &+\frac{1}{h_y^2} (\mathbf{{U}} \, \mathbf{{T}})_{ij} &= f_{i,j} \\
	(\mathbf{{T}} \, \mathbf{{U}})_{ij} &+\frac{h_x^2}{h_y^2} (\mathbf{{U}} \, \mathbf{{T}})_{ij} &= f_{i,j} h_x^2
\end{array}
\end{equation}

Because we cannot extract a common factor $\frac{1}{h^2}$ out of the left expression, we cannot move $\frac{1}{h^2}$ to the right side of the equation, unless $h_x = h_y$. Let $\frac{h_x^2}{h_y^2} = \beta$:

\begin{equation}
(\mathbf{{T}} \, \mathbf{{U}})_{ij} +\beta (\mathbf{{U}} \, \mathbf{{T}})_{ij} = f_{i,j} h_x^2
\end{equation}

This equation may be solved as the unit square poisson 2D case, given the eigenvalues. We may use the sine transform, but need a different solution because we are no longer on the unit square. In the 2D case, we may use 

\begin{align}
  \mathbf{u}_j^x(x) &= \sin(\frac{j \pi x}{L_x}) \\
  \mathbf{u}_j^y(y) &= \sin(\frac{j \pi y}{L_y})
\end{align}

Because the lengths vary between x and y, we have to find different eigenvalues to diagonalize the contributions from the $x$ and $y$ contributions to the equation system.
